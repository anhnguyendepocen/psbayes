\documentclass{slides}
\usepackage{xspace,tikz,graphicx,nth}

\title{A Brief History of Bayesian Inference}
\author[Andrews]{Mark Andrews}
\date{}

\begin{document}
{
	\begin{frame}
		\titlepage
	\end{frame}
}


\begin{frame}
	\frametitle{Bayes's Theorem}
	\begin{itemize}

	\item  The origin of Bayesian inference can be traced to a posthumously published 1763 essay
		\emph{Towards Solving a Problem in the Doctrine of Chances},
		written some years earlier by Reverend Thomas Bayes
		(1701-1761). 

	\item In this essay, Bayes focused on a problem that is mathematically
		identical to the problem of inferring the bias of a coin after
		observing the outcome of a sequence of flips of that coin.  
		
	\item Bayes showed that the probability that the coin's bias is exactly
		$\theta$ is proportional to the prior probability of $\theta$
		multiplied by the probability of the observed outcomes given
		$\theta$.
\end{itemize}
\end{frame}

\begin{frame}
	\frametitle{Laplace and Inverse Inference}
	\begin{itemize}
		\item Shortly after the publication of Bayes's essay,  the
			French polymath Pierre-Simon Laplace (1749-1827)
			independently rediscovered Bayes' theorem and began to
			apply it to practical problems of data analysis.


	\item Laplace was the first to present Bayesian inference in what is now its modern form:
		\[
			\underbrace{\Prob{\theta \given \data}}_{\text{Posterior}}
			= 
			\frac{\overbrace{\Prob{\data \given \theta}}^{\text{likelihood}}
				\overbrace{\Prob{\theta}}^{\text{prior}}}
				{\underbrace{\int \Prob{\data \given \theta}\Prob{\theta} d\theta}_{\text{marginal likelihood}}}
		\]

\end{itemize}
\end{frame}

\begin{frame}
	\frametitle{Frequentism}
	\begin{itemize}

		\item By the early \nth{20} century, a frequentist definition
			of probability became increasingly widely adopted.

		\item Crucially, frequentism entails that the concept of
			probability only applies to the outcomes of random
			physical processes.

\item As such, a parameter such as the bias of a coin, being a fixed but
	unknown quantity, can not be given a probabilistic interpretation.  

\item From this perspective, in the words of
R. A. Fisher, Bayesian methods were ``\ldots founded upon an error, and must be
wholly rejected.'' because
``(i)nferences respecting populations, from which known samples have been
drawn, cannot be expressed in terms of probability''. Fisher (1925).

	\end{itemize}
\end{frame}

\begin{frame}
	\frametitle{The Return of Bayesian Methods}
	
	\begin{itemize}
		\item Bayesian methods remained marginalized from the early to the late
		\nth{20} century
		\item However, their potential practical advantage over
		classical methods was that the could be applied in principle to any problem
		where a probabilistic model of data could be specified.  
		\item When computer power was minimal, any ``in principle'' advantages of Bayesian
		methods were not important.
		\item However, the roughly exponential growth of
		computational power from the 1970s onwards was the eventual catalyst for the adoption of
		Bayesian methods.
		\item By the late 1980's, Bayesian methods began returning to widespread use in science.
	\end{itemize}
\end{frame}
\end{document}
